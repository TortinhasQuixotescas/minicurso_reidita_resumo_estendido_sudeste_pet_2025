\section{Conclusão}%
\label{sec:conclusao}

Este trabalho apresentou um resumo estendido acerca da elaboração e aplicação de um minicurso de introdução à \gls{pw}, intitulado ``\gls{titulo_do_minicurso}''.
Seu objetivo é desenvolver um jogo do gênero \gls{ic} utilizando a filosofia de componentização da biblioteca React.

Algumas tecnologias específicas abordadas foram elementos de interface de \gls{html}, definição de estados e propriedades de componentes com \textit{useState}, efeitos colaterais com \textit{useEffect}, e a definição de tipos com \gls{ts}.

Acreditamos que o minicurso foi bem-sucedido em sua proposta de introduzir os conceitos básicos de desenvolvimento Web com React.
Recebemos feedbacks positivos acerca do aprofundamento atingido em um curto espaço de tempo, e da facilidade de compreensão do material de apoio.

Como limitações, ainda não foi possível avaliar o impacto do minicurso na formação dos alunos, uma vez que a aplicação do mesmo ocorreu em eventos pontuais.
Pretendemos aplicá-lo em um evento organizado pelo \gls{getsi} com mais tempo disponível, a fim de avaliar a eficácia do material em um contexto mais longo e recolher métricas do público.
