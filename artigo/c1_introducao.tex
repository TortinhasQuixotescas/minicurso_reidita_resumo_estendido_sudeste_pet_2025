\section{Introdução}%
\label{sec:introducao}

Desde a antiguidade, os jogos têm acompanhado a evolução da humanidade, se integrando à cultura e adaptando-se à tecnologia do tempo presente, conforme destacado por Huizinga em sua obra \textit{Homo Ludens}~\cite{Huizinga1938}.
Essas atividades, muitas vezes vistas simplesmente como formas de entretenimento, têm um poder surpreendente de ensino e desenvolvimento de competências, como habilidades de resolução de problemas, tomada de decisões e colaboração~\cite{gee2003what}.

Com o avanço da tecnologia digital, os jogos passaram a ocupar um espaço ainda mais relevante no campo educacional.
A utilização de jogos digitais como recursos pedagógicos tem se expandido rapidamente, oferecendo novas possibilidades para engajar estudantes e facilitar o processo de ensino-aprendizagem~\cite{gee2003what}.
Diante desse cenário, surge a oportunidade de explorar o potencial educativo dos jogos.

Após um período de inatividade decorrente da pandemia de 2020, o \gls{getsi} recebeu novos membros, totalizando cinco adesões no primeiro semestre do ano de 2022.
Esses membros se distribuíram entre alguns projetos, com destaque para aqueles de desenvolvimento web, os quais naturalmente se encaixam na proposta do curso de \gls{bsi}.

Posteriormente, foi oferecida ao grupo a oportunidade de ministrar um minicurso sobre o tema durante a Semana da Computação, evento que ocorre integrando a Semana do \gls{ice} da \gls{ufjf}.
O objetivo do curso é ensinar funcionalidades básicas e fundamentais acerca da estrutura de páginas Web na abordagem \gls{fe} (ligada à interface de usuário) assim como explorar tecnologias utilizadas pelo \gls{getsi} em seus projetos.
Dessa forma, os autores deste trabalho, desenvolveram o projeto intitulado ``\gls{titulo_do_minicurso}''.

Este relato apresenta a experiência do desenvolvimento e de aplicação do minicurso descrito.
O artigo está organizado da seguinte forma: a \autoref{sec:metodologia} descreve a metodologia adotada para o desenvolvimento do minicurso, incluindo as tecnologias abordadas e o material de apoio.
A \autoref{sec:resultados} apresenta os resultados obtidos com a aplicação do minicurso, incluindo a evolução do projeto.
Finalmente, a \autoref{sec:conclusao} apresenta as considerações finais e sugestões para trabalhos futuros.
