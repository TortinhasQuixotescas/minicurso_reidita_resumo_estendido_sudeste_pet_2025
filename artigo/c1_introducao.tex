\section{Introdução}%
\label{sec:introducao}

\todo[color=red!30]{A introdução foi baseada em um artigo de jogo educacional que fiz com o Igor, ela não explica bem a escolha desta motivação para o minicurso.}
Desde a antiguidade, os jogos têm acompanhado a evolução da humanidade, se integrando à cultura e adaptando-se à tecnologia do tempo presente, conforme destacado por Huizinga em sua obra \textit{Homo Ludens} \cite{huizinga1971homo}. Essas atividades, muitas vezes vistas simplesmente como formas de entretenimento, têm um poder surpreendente de ensino e desenvolvimento de competências, como habilidades de resolução de problemas, tomada de decisões e colaboração \cite{gee2003what}.

Nos últimos anos, com o avanço da tecnologia digital, os jogos passaram a ocupar um espaço ainda mais relevante no campo educacional. A utilização de jogos digitais como recursos pedagógicos tem se expandido rapidamente, oferecendo novas possibilidades para engajar estudantes e facilitar o processo de ensino-aprendizagem \cite{gee2003what}. Diante desse cenário, surge a oportunidade de explorar o potencial educativo dos jogos.

Neste sentido, os membros do GetSi, autores deste trabalho, propuseram-se a utilizar o desenvolvimento de um jogo como atividade prática e envolvente durante um minicurso de \textit{frontend} intitulado ``Começando no React com um jogo Idle Clicker''. Este relato apresenta a experiência do desenvolvimento desse projeto.

\subsection{Cultura do GetSi com Aplicações Web e Desenvolvimento Tecnológico}

Durante a pandemia, o GetSi passou por um período de inatividade, contando apenas com o membro tutor, o professor Igor Oliveira Knop. No entanto, em 2022, o grupo voltou a receber novos membros, totalizando cinco novas adesões no primeiro semestre. Esses membros se distribuíram entre alguns projetos, com destaque para aqueles de desenvolvimento web, que atraíram mais do que a metade dos membros.

Ao longo dos anos subsequentes, as aplicações web se tornaram recorrentes e uma das abordagens favoritas entre os membros do grupo. Neste passo, os membros adquirem experiência e se tornam capazes de desenvolver projetos cada vez mais complexos.

Após aproximadamente um ano de prática nessa seara, nos foi oferecida a oportunidade de ministrar um minicurso sobre o tema durante a Semana da Computação, evento que ocorre integrando a Semana do Instituto de Ciências Exatas da Universidade Federal de Juiz de Fora. O objetivo é ensinar funcionalidades básicas e fundamentais tanto da estrutura de páginas web \textit{frontend} (lado do cliente) quanto das tecnologias utilizadas pelo grupo em seus projetos.
