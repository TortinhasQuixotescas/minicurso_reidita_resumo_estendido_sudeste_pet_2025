\section{Metodologia}%
\label{sec:metodologia}

A fim de motivar os discentes no aprendizado de \gls{pw}, decidimos abordar o desenvolvimento de um jogo eletrônico.
Em razão do planejamento de duração de quatro horas, deveríamos selecionar um estilo de jogo cujas mecânicas permitissem a aplicação dos conceitos almejados em um ambiente de simples configuração e que fosse facilmente compreendido por um público diverso.

Dentre as opções, optamos pelo gênero \gls{ic}, ou também chamado de incremental.
Segundo~\citeonline{alharthi:2018:playing_to_wait}, esse estilo engloba experiências minimalistas, tipicamente disponíveis em navegadores Web, e que requerem pouca interação do jogador.
Sua mecânica central é realizar alguma ação repetitiva, como clicar em um botão, para acumular recursos e progredir no jogo, desbloqueando novas funcionalidades e desafios.
Alguns exemplos que inspiraram o desenvolvimento do jogo proposto são Cookie Clicker, de~\citeonline{orteil:2013:cookie_clicker} e Candy Box 2, de~\citeonline{aniwey:2013:candy_box_2}.

\subsection{As Tecnologias Abordadas}

A construção de páginas Web em navegadores é comumente realizada por meio da linguagem de marcação \gls{html}, da linguagem de estilização \gls{css} e da linguagem de programação \gls{js}.
Essa última é responsável por tornar a página dinâmica, ao possibilitar a manipulação dinâmica de elementos \gls{html}, o gerenciamento eficiente de eventos interativos e a modificação de estilos em tempo real com desempenho adequado às necessidades contemporâneas do desenvolvimento Web.

Seu caráter interpretado e sua sintaxe simplificada a tornam uma linguagem de fácil aprendizado, apresentando uma curva de aprendizado acessível para iniciantes.
Apesar disso, limitações dessa levaram à criação de ferramentas auxiliares, como o \textit{superset} \gls{ts}~\cite{microsoft:2012:typescript}, que adiciona tipagem estática e outros recursos à linguagem.

Acerca do método de manipulação da interface, a abordagem tradicional é a utilização de \gls{dom}, que é uma representação em árvore dos elementos da página~\cite{whatwg:2025:dom_standard}.
O desenvolvedor seleciona e manipula esses elementos por meio de métodos e propriedades disponíveis no \gls{js}, o que pode tornar o código complexo e de difícil manutenção.

A fim de reformar essa perspectiva, a biblioteca React gerencia o ciclo de atualização dos elementos em um \gls{dom} virtual.
O programador pode declarar estados mutáveis na aplicação, que serão automaticamente atualizados na interface quando houver mudanças.
Isso permite o desenvolvimento de componentes reutilizáveis, que favorecem a criação de aplicações \gls{spa}~\cite{meta:2025:thinking_in_react}.
Assim, é possível obter melhor organização, manutenibilidade e escalabilidade, fatores essenciais para a melhoria da qualidade do software.

Portanto, a linguagem JavaScript foi escolhida em integração com à biblioteca React, duas tecnologias amplamente utilizadas no desenvolvimento de aplicações modernas. Por fim, devido à limitação do tempo disponível, decidimos não abordar \textit{Cascade Style Sheets} (CSS), de forma que todo o estilo aplicado durante o minicurso é disponibilizado no material de apoio.
Com esse foco, o minicurso, voltado para introduzir os primeiros passos no \textit{React}, prioriza a componentização de elementos e a comunicação entre eles por meio da passagem de propriedades \textit{props}.

Além destes tópicos, o minicurso aborda dois principais \textit{React Hooks}, o \textit{useState} e o \textit{useEffect}.
Os \textit{hooks} são, desde 2019, uma das principais funcionalidades da biblioteca \textit{React}, permitindo a utilização de estado e outros recursos sem a necessidade de escrever classes~\cite{meta:2025:react_dom_hooks}. Eles tornam os componentes mais simples, reutilizáveis e legíveis.

O \textit{hook} \textit{useState} permite que componentes funcionais mantenham estado interno.
A função \textit{useState} recebe um valor inicial como argumento e retorna um vetor com dois elementos: o valor atual do estado, e uma função para atualizá-lo.
No escopo do curso, essa lógica é utilizada para exibir a quantidade disponível de cada recurso do jogo, que é atualizada a cada \textit{click} do usuário.

Já o \textit{hook} \textit{useEffect} permite que componentes realizem efeitos colaterais em resposta a mudanças externas, como chamadas a \textit{APIs}.
Durante o curso, este \textit{hook} é utilizado para atualizar o estado do jogo a cada segundo --- o que utiliza um processo do navegador ---, implementando produção constante de recursos, característica do caráter incremental do jogo.
