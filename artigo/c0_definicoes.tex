% =====
% Definições para gerar os artefatos
% =====

% -----
% Título 
% -----

% Título principal do trabalho. Parâmetros: título.
\DefineTitulo{Criando um jogo Idle Clicker}

% Subtítulo do trabalho (opcional). Parâmetros: subtítulo.
\DefineSubtitulo{Minicurso de Programação Web com React}

% Título principal do trabalho em língua estrangeira (opcional). Parâmetros: título.
\DefineTituloEstrangeiro{Creating an Idle Clicker game}

% Subtítulo do trabalho em língua estrangeira (opcional). Parâmetros: subtítulo.
\DefineSubtituloEstrangeiro{Workshop on Web Programming with React}

% -----

% -----
% Datas
% -----

% Data de submissão. Parâmetros: dia; mês por extenso, em letras minúsculas; ano no formato 1000.
\DefineDataDeSubmissao{23}{março}{2025}

% Data de aprovação. Parâmetros: dia; mês por extenso, em letras minúsculas; ano no formato 1000.
\DefineDataDeAprovacao{23}{março}{2025}

% -----

% -----
% Acesso
% -----

% Informações de acesso ao trabalho (opcional). Parâmetros: texto.
% \DefineInformacoesDeAcesso{DOI 10.0000/0000-0000.}

% -----

% -----
% Pessoas 
% -----

% Autor. Parâmetros: Último sobrenome; restante do nome.
\DefineAutor{Malosto}{Celso Gabriel}{Grupo GET Sistemas de Informação (GetSi), Universidade Federal de Juiz de Fora (UFJF), Campus Juiz de Fora. E-mail: gabriel.malosto@estudante.ufjf.br.}
\DefineAutor{Santos}{Lucas Paiva}{Grupo GET Sistemas de Informação (GetSi), Universidade Federal de Juiz de Fora (UFJF), Campus Juiz de Fora. E-mail: lucas.paiva@estudante.ufjf.br.}

% -----

% =====
