% =====
% Conteúdo dos artefatos
% =====

% -----
% Resumo (Obrigatório)
% -----

% --- Língua vernácula ---

% Palavras chave. Parâmetros: palavra-chave um; palavra-chave dois; palavra-chave três; palavra-chave quatro (opcional); palavra-chave cinco (opcional). Deixe em branco os campos opcionais que não deseja preencher.
\DefinePalavrasChave{minicurso}{programação}{web}{react}{jogo incremental}

\insereResumo{%
    Este resumo expandido apresenta o minicurso ``\gls{titulo_do_minicurso}'', elaborado pelo \glsxtrshort{getsi} da \glsxtrshort{ufjf}.
    Seu objetivo é apresentar conceitos básicos de \gls{pw} utilizando a biblioteca React.
    Para tal, desenvolve-se um jogo do gênero incremental para navegador, em que o usuário interage com elementos na tela para acumular recursos.
    Explora-se a criação de componentes semânticos, a manipulação de estados e a resposta a eventos.
    Um livro digital foi elaborado para guiar os participantes durante aplicações do curso em eventos na universidade.
    Os feedbacks foram positivos, havendo preocupações quanto ao limite de tempo.
    Ainda resta realizar uma análise quantitativa para avaliar o aproveitamento do conteúdo pelos alunos.
}

% --- Língua estrangeira ---

% Parâmetros: língua (english, french, spanish, italian, german, dutch); palavras-chave; conteúdo.

% Comente para remover o resumo em língua estrangeira.
\insereResumoEmLinguaEstrangeira%
{english}
{\formataPalavrasChave{workshop}{programmming}{web}{react}{incremental game}}
{%
    This expanded abstract presents the workshop ``\gls{titulo_do_minicurso}'', developed by the \glsxtrshort{getsi} of \glsxtrshort{ufjf}.
    Its goal is to introduce basic concepts of \gls{pw} using the React library.
    To do so, an incremental game is developed for the browser, in which the user interacts with elements on the screen to accumulate resources.
    It explores the creation of semantic components, state manipulation, and event handling.
    A digital book was created to guide participants during course applications at university events.
    Feedback was positive, with concerns about the time limit.
    A quantitative analysis is still needed to evaluate the students' understanding of the content.
}

% -----

% -----
% Agradecimentos (Opcional)
% -----

% Comente para remover os agradecimentos.
% \insereAgradecimentos{%
%     Agradecemos aos professores Igor de Oliveira Knop e Marcelo Caniato Renhe pela tutoria do \gls{getsi}, que nos apoiaram na realização do minicurso e na elaboração deste trabalho.
% }

% -----

% -----
% Glossário (Opcional)
% -----

% Comente para remover o glossário.
% \insereGlossario{}

% -----

% =====
