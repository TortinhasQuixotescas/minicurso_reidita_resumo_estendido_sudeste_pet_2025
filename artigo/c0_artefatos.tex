% =====
% Conteúdo dos artefatos
% =====

% -----
% Resumo (Obrigatório)
% -----

% --- Língua vernácula ---

% Palavras chave. Parâmetros: palavra-chave um; palavra-chave dois; palavra-chave três; palavra-chave quatro (opcional); palavra-chave cinco (opcional). Deixe em branco os campos opcionais que não deseja preencher.
\DefinePalavrasChave{minicurso}{computação}{programação web}{react}{}

\insereResumo{%
    \lipsum[1]
}

% --- Língua estrangeira ---

% Parâmetros: língua (english, french, spanish, italian, german, dutch); palavras-chave; conteúdo.

% Comente para remover o resumo em língua estrangeira.
\insereResumoEmLinguaEstrangeira%
{english}
{\formataPalavrasChave{workshop}{computing}{web programming}{react}{}}
{%
    \lipsum[2]
}

% -----

% -----
% Agradecimentos (Opcional)
% -----

% Comente para remover os agradecimentos.
% \insereAgradecimentos{%
%     Agradecemos aos professores Igor de Oliveira Knop e Marcelo Caniato Renhe pela tutoria do \gls{getsi}, que nos apoiaram na realização do minicurso e na elaboração deste trabalho.
% }

% -----

% -----
% Glossário (Opcional)
% -----

% Comente para remover o glossário.
% \insereGlossario{}

% -----

% =====
